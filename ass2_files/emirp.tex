\documentclass[a4paper,12pt,fleqn]{scrartcl}
\usepackage[l2tabu,orthodox]{nag}% Old habits die hard. All the same, there are commands, classes and packages which are outdated and superseded. nag provides routines to warn the user about the use of those.
\usepackage[all,error]{onlyamsmath}% Error on deprecated math commands like $$ $$.
\usepackage{fixltx2e}
\usepackage[strict=true]{csquotes}
\usepackage{listings}
\lstset{frame=single,framerule=0pt,language={C},showstringspaces=false,numbers=left,columns=fullflexible}
\usepackage{color}
\usepackage{2111defs2,2111theorems}
\usepackage{nicefrac}


\newcommand{\singlequote}[1]{`#1'}
\newcommand{\myjustification}[2][\Equiv]{{}#1{} &{\bcolor\langle\text{#2}\rangle}\\}
\newcommand{\remark}[1]{{\sffamily\color{blue}{#1}}}
\newcommand{\myImplies}[2]{\color{red}#1\color{blue}\Implies\color{red}#2\color{black}}
\newcommand{\pre}{\mathit{pre}}
\newcommand{\post}{\mathit{post}}
\newcommand{\emirp}{\textsc{emirp}\xspace}


\title{Assignment 2 - Emirps}
\date{COMP2111 18s1}
\author{Rahil Agrawal z5165505\\Aditya Karia zXXXXXXX}

\begin{document}
\pagenumbering{gobble}
\maketitle
\pagenumbering{Roman}

\section{Task 1 - Specification Statement}
\label{sec:task-1}
\remark{Notes:\\-Write neatly\\-make sure grammar is correct\\-look at examples for default spec structure.}\\\\
\remark{Define an Emirp using 2 functions - reverse and prime. Make these functions match with their given specs in order to help prove implications.}\\\\
Pre condition: $n$ is a positive number - $n>0$\\\\
Post condition $EMIRP(r,n)$ where $r$ is the $n^{th}$ emirp(where emirp is as defined above).
Therefore our program can be specified by:\\
$\PROC~\emirp(\VALUE~n, \RESULT~r)\cdot{}\\
 \qquad  \nt{n,r:\left[
    \begin{array}{l}
      n>0, Emirp(r,n)
    \end{array}
  \right]}{(1)}\\$

\section{Task 2 - Derivation}
\label{sec:task-1}
\begin{align*}
  &\PROC~\emirp(\VALUE~n, \RESULT~r)\cdot{}\\
  &\qquad  \nt{n,r:\left[
    \begin{array}{l}
      n>0, Emirp(r,n)
    \end{array}
  \right]}{(1)}\\
% 
  \lrefstep{(1)}
  {\textbf{c-frame}}
  {
  \nt{r:\left[
    \begin{array}{l}
      n>0, Emirp(r,n)
    \end{array}
  \right]}{(2)}}\\
%
  \lrefstep{(2)}
  {\textbf{i-loc}}
  {
  \nt{i,r:\left[
    \begin{array}{l}
      n>0, Emirp(r,n)
    \end{array}
  \right]}{(3)}}\\
%
  \lrefstep{(3)}
  {\textbf{seq}}
  {
  \nt{i,r:\left[
    \begin{array}{l}
      n>0, i=0 \And n>0
    \end{array}
  \right]}{(4)};\\
  &\nt{i:\left[
    \begin{array}{l}
      i=0 \And n>0, Emirp(r,n)
    \end{array}
  \right]}{(5)}}\\
%
  \lrefstep{(4)}
  {\textbf{c-frame}}
  {
  \nt{i:\left[
    \begin{array}{l}
      n>0, i=1 \And n>0
    \end{array}
  \right]}{}}\\
%
  \refstep{\textbf{ass - \color{blue}(1)}}
  {i\Ass 1}\\
%
  \lrefstep{(5)}
  {\textbf{seq}}
  {
  \nt{i,r:\left[
    \begin{array}{l}
      i=1 \And n>0, Inv
    \end{array}
  \right]}{(6)};\\
  &\nt{i,r:\left[
    \begin{array}{l}
      Inv, Inv \And i=n
    \end{array}
  \right]}{(7)};\\
  &\nt{i,r:\left[
    \begin{array}{l}
      Inv \And i=n, Emirp(r,n)
    \end{array}
  \right]}{(8)}}\\
%
  \lrefstep{(6)}
  {\textbf{w-pre, c-frame - \color{blue}(2)}}
  {
  \nt{r:\left[
    \begin{array}{l}
      Inv\subst{13}{r}, Inv
    \end{array}
  \right]}{(9)}}\\
%
  \refstep{\textbf{ass - \color{blue}(3)}}
  {r\Ass 13}\\
%
  \lrefstep{(7)}
  {\textbf{while}}
  {
  \WHILE~i\neq n~\DO\\
  &\qquad \nt{i,r:\left[
    \begin{array}{l}
      Inv\And i\neq n, Inv
    \end{array}
  \right]}{(10)}\\
  &\OD;}\\
%
  \lrefstep{(10)}
  {\textbf{seq}}
  {
  \nt{r:\left[
    \begin{array}{l}
      Inv\And i\neq n, Inv\subst{r+1}{r}
    \end{array}
  \right]}{(11)};\\
  &\nt{r:\left[
    \begin{array}{l}
      Inv\subst{r+1}{r}, Inv
    \end{array}
  \right]}{(12)}}\\
\end{align*}
\begin{align*}
  \lrefstep{(11)}
  {\textbf{i-loc}}
  {
  \nt{a,i,r:\left[
    \begin{array}{l}
      Inv\subst{r+1}{r}, Inv
    \end{array}
  \right]}{(13)}}\\
%
  \lrefstep{(13)}
  {\textbf{seq}}
  {
  \nt{a,i,r:\left[
    \begin{array}{l}
      Inv\subst{r+1}{r}, Inv\subst{r+1}{r}\And a=1
    \end{array}
  \right]}{(14)};\\
  &\nt{a,i,r:\left[
    \begin{array}{l}
      Inv\subst{r+1}{r}\And a=1, Inv
    \end{array}
  \right]}{(15)}
  }\\
%
  \lrefstep{(14)}
  {\textbf{c-frame}}
  {
  \nt{a:\left[
    \begin{array}{l}
      Inv\subst{r+1}{r}, Inv\subst{r+1}{r}\And a=1  
    \end{array}
  \right]}{(16)}}\\
%
  \refstep{\textbf{ass - \color{blue}(4)}}
  {a\Ass 1}\\
%
  \lrefstep{(15)}
  {\textbf{seq}}
  {
  \nt{a,i,r:\left[
    \begin{array}{l}
      Inv\subst{r+1}{r}\And a=1 ,(a=1 \text{ }\And \not\Exi{k \in 2..(r-1)}{r\textbf{ mod }k=0})\\
      \Or(a=0 \And \Exi{k \in 2..(r-1)}{r\textbf{ mod }k=0}))
    \end{array}
  \right]}{(17)};\\
  &\nt{a,i,r:\left[
    \begin{array}{l}
      (a=1 \text{ }\And \not\Exi{k \in 2..(r-1)}{r\textbf{ mod }k=0})\\\Or(a=0 \And \Exi{k \in 2..(r-1)}{r\textbf{ mod }k=0})),
      Inv
    \end{array}
  \right]}{(18)}
  }\\
%
  \lrefstep{(18)}{\textbf{if}}
  {\IF~a=1\\
  &\THEN~\nt{a,i,r:[a=1\And\pre(18),\post(18)]}{(19)}\\
  &\ELSE~\nt{p:[a\neq1\And\pre(18),\post(18)]}{(20)}\\
  &\FI}\\
%
\end{align*}
We gather the code for the procedure body of $\emirp$:
\begin{tabbing}% unfortunately, tabbing doesn't do maths

\end{tabbing}

\section{Task 3 - C Code}
\label{sec:task-1}
\lstinputlisting{emirp.c}
\remark{- Write something about how the C code relates.\\- Compare with examples}
\end{document}
