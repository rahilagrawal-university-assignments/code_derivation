\documentclass[a4paper,12pt,fleqn]{scrartcl}
\usepackage[l2tabu,orthodox]{nag}% Old habits die hard. All the same, there are commands, classes and packages which are outdated and superseded. nag provides routines to warn the user about the use of those.
\usepackage[all,error]{onlyamsmath}% Error on deprecated math commands like $$ $$.
\usepackage{fixltx2e}
\usepackage[strict=true]{csquotes}
\usepackage{listings}
\lstset{frame=single,framerule=0pt,language={C},showstringspaces=false,numbers=left,columns=fullflexible}
%\usepackage{color}
\usepackage[usenames, dvipsnames]{color}
\usepackage{2111defs2,2111theorems}
\usepackage{nicefrac}

\newcommand{\var}{\textrm{$var$ }}
\newcommand{\singlequote}[1]{`#1'}
\newcommand{\myjustification}[2][\Equiv]{{}#1{} &{\bcolor\langle\text{#2}\rangle}\\}
\newcommand{\remark}[1]{{\sffamily\color{blue}{#1}}}
\newcommand{\myImplies}[2]{\color{red}#1\color{blue}\Implies\color{red}#2\color{black}}
\newcommand{\pre}{\mathit{pre}}
\newcommand{\post}{\mathit{post}}
\newcommand{\emirp}{\textsc{emirp}\xspace}
\newcommand{\isPrime}{\textsc{isPrime}\xspace}
\newcommand{\myCode}[1]{\mathbf{\color{ProcessBlue}#1}}
\newcommand{\myRefines}[2]{\color{red}{#1}\color{black}\isrefinedby\color{red}{#2}}

\title{Assignment 2 - Emirps}
\date{COMP2111 18s1}
\author{Rahil Agrawal z5165505\\Aditya Karia z5163287}

\begin{document}
\pagenumbering{gobble}
\maketitle
\pagenumbering{Roman}

\section{Task 1 - Specification Statement}
\label{sec:task-1}
A prime number is a positive integer that is only divisible by 1 and itself. Therefore, we can say that a number $r$ is prime if it is not divisible by any number between 2 and $r-1$ inclusive.\\\\
Therefore, we can define a primality check function as follows:
\begin{gather*}
  \color{blue}isPrime(r) = 
     \begin{cases}
       \text{true} &\quad \neg\Exi{k \in 2..(r-1)}{r\textbf{ mod }k=0}\\
       \text{false} &\quad  \Exi{k \in 2..(r-1)}{r\textbf{ mod }k=0}\\ 
     \end{cases}
\end{gather*}\\
The $reverse(r,s)$ function can be used to store the reverse of a number $r$ in a variable called $s$.\\\\
Having defined a primality check function $isPrime(r,a)$ and a function to store the reverse of a number $r$ in $s$, we define an $emirp$.\\\\
An $emirp$ is a prime number whose reversal is also prime, but which is not a palindromic prime.\\\\
Therefore, if $EMIRP(r,n)$ states that $r$ is the $n^{th}$ $emirp$,where $n$ is a positive integer, then:
\begin{gather*}
  \color{blue}EMIRP(r,n) = 
     \begin{cases}
       \text{true} &\quad {isPrime(r) \And reverse(r,s) \And isPrime(s) \And r\neq s}\\
       \text{false} &\quad  otherwise\\
     \end{cases}
\end{gather*}\\
We want to find the $n^{th}$ $emirp$. Having defined the limitations on $n$ ($n>0$) and described what it means for a $r$ to be the $n^{th}$, we can specify our program with by:\\
$\PROC~\emirp(\VALUE~n, \RESULT~r)\cdot{}
 \nt{n,r,x:\left[
    \begin{array}{l}
      n>0, Emirp(r,n)
    \end{array}
  \right]}{(1)}\\$
\pagebreak
\section{Task 2 - Derivation}
\label{sec:task-2}
\begin{align*}
  &\PROC~\emirp(\VALUE~n, \RESULT~r)\cdot{}	
  \nt{n,r,x:\left[
    \begin{array}{l}
      n>0, Emirp(r,n)
    \end{array}
  \right]}{(1)}\\
% 
  \lrefstep{(1)}
  {\textbf{c-frame}}
  {
  \nt{r,x:\left[
    \begin{array}{l}
      n>0, Emirp(r,n)
    \end{array}
  \right]}{(2)}}\\
%
  \lrefstep{(2)}
  {\textbf{i-loc}}
  {
  \nt{i,r,x:\left[
    \begin{array}{l}
      n>0, Emirp(r,n)
    \end{array}
  \right]}{(3)}}\\
%
  \lrefstep{(3)}
  {\textbf{seq}}
  {
  \nt{i,x,r:\left[
    \begin{array}{l}
      n>0, i=1 \And x=13 \And n>0
    \end{array}
  \right]}{(4)};\\
  &\nt{i,x:\left[
    \begin{array}{l}
      i=1 \And x=13 \And n>0, Emirp(r,n)
    \end{array}
  \right]}{(5)}}\\
%
  \lrefstep{(4)}
  {\textbf{c-frame}}
  {
  i,x:\left[
    \begin{array}{l}
      n>0, i=1 \And x=13 \And n>0
    \end{array}
  \right]}\\
%
  \refstep{\textbf{ass - \color{blue}(1)}}
  {\myCode{i\Ass 1}\\
  &\myCode{x\Ass 13}}\\
%
  \lrefstep{(5)}
  {\textbf{seq}}
  {
  \nt{i,r,x:\left[
    \begin{array}{l}
      i=1\And x=13 \And n>0, Inv
    \end{array}
  \right]}{(6)};\\
  &\nt{i,r,x:\left[
    \begin{array}{l}
      Inv, Inv \And i=n
    \end{array}
  \right]}{(7)};\\
  &\nt{i,r,x:\left[
    \begin{array}{l}
      Inv \And i=n, Emirp(r,n)
    \end{array}
  \right]}{(8)}}\\
%
  \lrefstep{(6)}
  {\textbf{w-pre, c-frame - \color{blue}(2)}}
  {r,x:\left[
    \begin{array}{l}
      Inv\subst{13}{r}, Inv
    \end{array}
  \right]}\\
%
  \refstep{\textbf{ass - \color{blue}(3)}}
  {\myCode{r\Ass 13}}\\
%
  \lrefstep{(7)}
  {\textbf{while}}
  {
  \myCode{\WHILE~i\neq n~\DO}\\
  &\qquad \nt{i,r,x:\left[
    \begin{array}{l}
      Inv\And i\neq n, Inv
    \end{array}
  \right]}{(9)}\\
  &\myCode{\OD;}}\\
%
  \lrefstep{(8)}
  {\textbf{w-pre, c-frame - \color{blue}(4) }}
  {
  \nt{r,x:\left[
    \begin{array}{l}
      \emirp(r,n), \emirp(r,n)
    \end{array}
  \right]}{(10)}}\\
%
  \refstep{\textbf{skip - \color{blue}(5)}}
  {\myCode{skip}}\\
%
  \lrefstep{(9)}
  {\textbf{seq}}
  {
  \nt{r,x:\left[
    \begin{array}{l}
      Inv\And i\neq n, Inv\subst{x+1}{x}
    \end{array}
  \right]}{(10)};\\
  &\nt{r,x:\left[
    \begin{array}{l}
      Inv\subst{x+1}{x}, Inv
    \end{array}
  \right]}{(11)}}\\
\end{align*}
\begin{align*}
  \lrefstep{(10)}
  {\textbf{ass - \color{blue}(6)}}
  {\myCode{x\Ass{x+1}}}\\
%
  \lrefstep{(11)}
  {\textbf{i-loc, seq}}
  {
  \nt{a,i,r,x:\left[
    \begin{array}{l}
      Inv\subst{x+1}{x}, Inv\subst{x+1}{x}\And a=1
    \end{array}
  \right]}{(12)};\\
  &\nt{a,i,r,x:\left[
    \begin{array}{l}
      Inv\subst{x+1}{x}\And a=1, Inv
    \end{array}
  \right]}{(13)}
  }\\
%
  \lrefstep{(12)}
  {\textbf{c-frame}}
  {
  a,x:\left[
    \begin{array}{l}
      Inv\subst{x+1}{x}, Inv\subst{x+1}{x}\And a=1  
    \end{array}
  \right]}\\
%
  \refstep{\textbf{ass - \color{blue}(7)}}
  {\myCode{a\Ass 1}}\\
%
  \lrefstep{(13)}
  {\textbf{seq}}
  {
  \nt{a,i,r,x:\left[
    \begin{array}{l}
      Inv\subst{x+1}{x}\And a=1 ,(a=1 \text{ }\And \neg\Exi{k \in 2..(x-1)}{x\textbf{ mod }k=0})\\
      \Or(a=0 \And \Exi{k \in 2..(x-1)}{x\textbf{ mod }k=0}))
    \end{array}
  \right]}{(14)};\\
  &\nt{a,i,r,x:\left[
    \begin{array}{l}
      (a=1 \text{ }\And \neg\Exi{k \in 2..(x-1)}{x\textbf{ mod }k=0})\\\Or(a=0 \And \Exi{k \in 2..(x-1)}{x\textbf{ mod }k=0})),
      Inv
    \end{array}
  \right]}{(15)}
  }\\
%
  \lrefstep{(14)}
  {\textbf{w-pre - \color{blue}(8)}}
  {
  a,x:\left[
    \begin{array}{l}
      a=1\And x>0, post(14)  
    \end{array}
  \right]}\\
%
  \refstep{\textbf{proc}}{
    \myCode{isPrime(x,a)}
  }\\
%
  \lrefstep{(15)}{\textbf{if}}
  {\myCode{\IF~a=1}\\
  &\myCode{\THEN}~\nt{a,i,r,x:[a=1\And\pre(15),\post(15)]}{(16)}\\
  &\myCode{\ELSE}~\nt{p,x:[a\neq1\And\pre(15),\post(15)]}{(17)}\\
  &\myCode{\FI}}\\
%
  \lrefstep{(16)}
  {\textbf{i-loc}}
  {
  a,i,r,s,x:\left[
    \begin{array}{l}
      \pre(16), \post(16)  
    \end{array}
  \right]}\\
%
  \refstep{\textbf{seq}}
  {
  \nt{s,x:\left[
    \begin{array}{l}
      \pre(16), s=0 \And x>0
    \end{array}
  \right]}{(18)};\\
  &\nt{a,i,r,s,x: \left[
    \begin{array}{l}
      x>0 \And s=0, \post(16)
    \end{array}
  \right]}{(19)}}\\
%
\lrefstep{(17)}{\textbf{c-frame, w-pre- \color{blue}(9)}}
  {
  i,r:\left[
    \begin{array}{l}
        Inv, Inv
    \end{array}\right]}\\
%
  \refstep{\textbf{skip - \color{blue}(10)}}
  {\myCode{skip}}\\
%
  \lrefstep{(18)}{\textbf{ass - \color{blue}(11)}}
  {\myCode{s\Ass 0}}\\  
\end{align*}
\begin{align*}
  \lrefstep{(19)}{\textbf{seq}}
  {
  \nt{s,x:\left[
    \begin{array}{l}
      \pre(19), s = \sum\limits_{i=0}^{c(x)}(S_i10^{i})
    \end{array}
  \right]}{(20)};\\
  &\nt{a,i,r,s,x: \left[
    \begin{array}{l}
      s = \sum\limits_{i=0}^{c(x)}(S_i10^{i}), \post(19)
    \end{array}
  \right]}{(21)}}\\
%
  \lrefstep{(20)}{\textbf{i-con, c-frame, w-pre - \color{blue}(12)}}
  {
  \mathbf{con}\text{ } S: [10]^{*}\cdot{} s:\left[
    \begin{array}{l}
      x = \sum\limits_{i=0}^{c(x)}(S_i10^{(c(x)-i)}) \And x>0, s = \sum\limits_{i=0}^{c(x)}(S_i10^{i})
    \end{array}
  \right]}\\
%
  \refstep{\textbf{proc}}
  {\myCode{reversen(x,s)}}\\
%
  \lrefstep{(21)}
  {\textbf{i-loc, seq}}
  {
  \nt{a,i,r,s,b,x:\left[
    \begin{array}{l}
      \pre(21), \pre(21)\And b=1  
    \end{array}
  \right]}{(22)};\\
  &\nt{a,i,r,s,b,x:\left[
    \begin{array}{l}
      \pre(21)\And b=1, \post(21)  
    \end{array}
  \right]}{(23)}}\\
%
  \lrefstep{(22)}{\textbf{c-frame, ass - \color{blue}(13)}}
   {
     \myCode{b\Ass 1} 
   }\\
%
  \lrefstep{(23)}{\textbf{seq}}
  {
  \nt{a,i,r,s,b,x:\left[
    \begin{array}{l}
        pre(21) \And b = 1, (b=1 \text{ }\And \neg\Exi{k \in 2..(s-1)}{s\textbf{ mod }k=0})\\
      \Or(b=0 \And \Exi{k \in 2..(s-1)}{s\textbf{ mod }k=0})
    \end{array}
  \right]}{(24)};\\
  &\nt{a,i,r,s,b,x:\left[
    \begin{array}{l}
        (b=1 \text{ }\And \neg\Exi{k \in 2..(s-1)}{s\textbf{ mod }k=0})\\
      \Or(b=0 \And \Exi{k \in 2..(s-1)}{s\textbf{ mod }k=0}) , post(21)
    \end{array}
  \right]}{(25)}}\\
%
  \lrefstep{(24)}{\textbf{w-pre - \color{blue}{(14)}}}
  {a,i,r,s,b,x:\left[
    \begin{array}{l}
        s>0 \And b=1, (b=1 \text{ }\And \neg\Exi{k \in 2..(s-1)}{s\textbf{ mod }k=0})\\
      \Or(b=0 \And \Exi{k \in 2..(s-1)}{s\textbf{ mod }k=0})
    \end{array}\right]}\\
%
  \refstep{\textbf{proc}}{
    \myCode{isPrime(s,b)}
  }\\
%
  \lrefstep{(25)}{\textbf{if}}
  {
  \myCode{\IF~b=1 \And s\neq x}\\
  &\myCode{\THEN}~\nt{i,x:[b=1\And s\neq x \And\pre(25),\post(25)]}{(26)}\\
  &\myCode{\ELSE}~\nt{i,r,a,s,b,x:[(b\neq1 \Or s=x)\And\pre(25),\post(25)]}{(27)}\\
  &\myCode{\FI;}
  }\\
%
  \lrefstep{(26)}{\textbf{c-frame, w-pre- \color{blue}(15)}}
  {
  a,i,r,s,b,x:\left[
    \begin{array}{l}
        Inv\subst{i+1}{i}\subst{x}{r}, Inv
    \end{array}\right]}\\
%
  \refstep{\textbf{ass - \color{blue}(16)}}
  {
    \myCode{i\Ass i+1}\\
    &\myCode{r\Ass x}\\
  }
\end{align*}
\begin{align*}
  \lrefstep{(27)}{\textbf{c-frame, w-pre- \color{blue}(17)}}
  {
  a,i,r,s,b,x:\left[
    \begin{array}{l}
        Inv, Inv
    \end{array}\right]}\\
%
  \refstep{\textbf{skip - \color{blue}(18)}}
  {
    \myCode{skip}
  }\\\\
%
  &\PROC~\isPrime(\VALUE~r, \RESULT~a)\cdot{}\\
  &\qquad\nt{r,a:\left[
    \begin{array}{l}
      a=1\And r>0, (a=1 \text{ }\And \neg\Exi{k \in 2..(r-1)}{r\textbf{ mod }k=0})\\
      \Or(a=0 \And \Exi{k \in 2..(r-1)}{r\textbf{ mod }k=0}
    \end{array}
  \right]}{(1)}\\
%
  \lrefstep{(1)}
  {\textbf{seq, i-loc}}
  {
  \nt{r,a,j:\left[
    \begin{array}{l}
      a=1\And r>0, a=1\And r>0 \And j=2 
    \end{array}
  \right]}{(2)};\\
  &\nt{r,a,j:\left[
    \begin{array}{l}
      a=1\And r>0 \And j=2, (a=1 \text{ }\And \neg\Exi{k \in 2..(r-1)}{r\textbf{ mod }k=0})\\
      \Or(a=0 \And \Exi{k \in 2..(r-1)}{r\textbf{ mod }k=0}
    \end{array}
  \right]}{(3)}}\\
%
  \lrefstep{(2)}
  {\textbf{ass - \color{blue}(19)}}
  {\myCode{j \Ass 2}}\\
%
  \lrefstep{(3)}
  {\textbf{seq}}
  {
  \nt{r,a,j:\left[
    \begin{array}{l}
      a=1\And r>0 \And j=2, Inv_{2} 
    \end{array}
  \right]}{(4)};\\
  &\nt{r,a,j:\left[
    \begin{array}{l}
      Inv_{2}, Inv_{2} \And j=r 
    \end{array}
  \right]}{(5)};\\
  &\nt{r,a,j:\left[
    \begin{array}{l}
      Inv_{2} \And j=r, post(3)
    \end{array}
  \right]}{(6)}}\\
%
  \lrefstep{(4)}
  {\textbf{w-pre - \color{blue}(20)}}
  {
  r,a,j:\left[
    \begin{array}{l}
      Inv_{2}, Inv_{2} 
    \end{array}
  \right]}\\
%
  \refstep{\textbf{skip - \color{blue}(21)}}{
    \myCode{skip}
  }\\
%
  \lrefstep{(6)}
  {\textbf{w-pre - \color{blue}(22)}}
  \nt{r,a,j:\left[
    \begin{array}{l}
      post(3), post(3) 
    \end{array}
  \right]}{(7)}\\
%
  \refstep{\textbf{skip - \color{blue}(23)}}{
    \myCode{skip}
  }\\
%
  \lrefstep{(5)}
  {\textbf{while}}
  {
  \myCode{\WHILE~j\neq r~\DO}\\
  &\qquad \nt{r, j:\left[
    \begin{array}{l}
      Inv_{2}\And j\neq r, Inv_{2}
    \end{array}
  \right]}{(8)}\\
  &\myCode{\OD;}}\\
%
  \lrefstep{(8)}
  {\textbf{seq}}
  {
  \nt{r,j:\left[
    \begin{array}{l}
      pre(8), Inv_{2}\subst{j+1}{j} 
    \end{array}
  \right]}{(9)};\\
  &\nt{r,j:\left[
    \begin{array}{l}
      Inv_{2}\subst{j+1}{j}, Inv_{2} 
    \end{array}
  \right]}{(10)}}\\
\end{align*}
\begin{align*}
  \lrefstep{(9)}{\textbf{if}}
  {\myCode{\IF~r\text{ }\textbf{mod}\text{ }j = 0}\\
  &\myCode{\THEN}~\nt{a:[r\text{ }\textbf{mod}\text{ }j=0\And\pre(9),\post(9)]}{(11)}\\
  &\myCode{\ELSE}~\nt{a:[r\text{ }\textbf{mod}\text{ }j\neq0\And\pre(9),\post(9)]}{(12)}\\
  &\myCode{\FI;}}\\
%
  \lrefstep{(10)}
  {\textbf{ass - \color{blue}{(24)}}}
  {\myCode{j \Ass j+1}}\\
  \lrefstep{(11)}
  {\textbf{w-pre - \color{blue}(25)}}
  {r,a,j:\left[
    \begin{array}{l}
      Inv_{2}\subst{j+1}{j}\subst{0}{a}, post(11)
    \end{array}
  \right]}\\
%  
  \refstep{\textbf{ass - \color{blue}{(26)}}}{
    \myCode{a \Ass 0}
  }\\
  \lrefstep{(12)}
  {\textbf{w-pre - \color{blue}(27)}}
  {r,a,j:\left[
    \begin{array}{l}
      Inv_{2}\subst{j+1}{j}, post(11)
    \end{array}
  \right]}\\
%
  \refstep{\textbf{skip - \color{blue}{(28)}}}
  {
    \myCode{skip}
  }
\end{align*}\\
We gather the code for the procedure body of $\emirp$:
\begin{gather*}
  \mathbf{EMIRP(r,n):}\\
  \qquad \var i\Ass 1;\\
  \qquad \var x\Ass 13;\\
  \qquad r\Ass 13;\\
  \qquad \WHILE~j\neq r~\DO\\
  \qquad \qquad x\Ass {x+1};\\
  \qquad \qquad \var a\Ass 1;\\
  \qquad \qquad isPrime(x,a);\\
  \qquad \qquad \IF~a=1~\THEN\\
  \qquad \qquad \qquad \var s\Ass 0;\\
  \qquad \qquad \qquad reversen(x,s);\\
  \qquad \qquad \qquad \var b\Ass 1;\\
  \qquad \qquad \qquad isPrime(s,b);\\
  \qquad \qquad \qquad \IF~b=1\And s\neq x~\THEN\\
  \qquad \qquad \qquad \qquad i\Ass {i+1};\\
  \qquad \qquad \qquad \qquad r\Ass x;\\
  \qquad \OD;
\end{gather*}
Also, we gather the code for the procedure body of $\isPrime$:
\begin{gather*}
  \mathbf{isPrime(r,j):}\\
  \qquad \var j\Ass 2;\\
  \qquad \WHILE~j\neq r~\DO\\
  \qquad \qquad \IF~(r \textrm{ mod }j)=0~\THEN\\
  \qquad \qquad \qquad a\Ass 0;\\
  \qquad \qquad j\Ass {j+1};\\
  \qquad \OD;\\
\end{gather*}
We have derived our code. However we need to prove \textbf{\color{blue}some }\color{black} refinements.

\subsection{\color{blue}Implication 1\color{black}: $\myRefines{(4)}{i\Ass 1}$}
\begin{align*}
&\text{To prove: } i = i_{0} \And n > 0 \Implies (i=1 \And x=13 \And n>0)\subst{1}{i}\subst{13}{x}\\\\
%
&\text{Proof:}\\
%
&LHS =  i = i_{0} \And n > 0 \\
%
\myjustification[\Implies]{1=1 $\And$ 13=13 is vacuously true}
&1=1 \And 13=13 \And i=i_{0} \And n>0\\
%
\myjustification[\Implies]{$A \And B \And C \And D\Implies A\And B \And C$}
&1=1 \And 13=13 \And n>0\\
%
\myjustification[\Implies]{$1=1 \Implies (i=1)\subst{1}{i}, 13=13 \Implies (x=13)\subst{13}{x}$}
&(i=1\And x=13\And n>0)\subst{1}{i}\subst{13}{x}\\
%
\myjustification[\Implies]{clearly}
&RHS
\end{align*}

\subsection{\color{blue}Implication 2\color{black}: $\myRefines{(6)}{r,x: [Inv\subst{13}{x}, Inv]}$}
To prove w-pre we need to prove: $pre \Implies pre^{'}$
\begin{align*}
&\text{To prove: } i=1 \And n>0 \And x=13 \Implies Inv\subst{13}{r}\\\\
%
&\text{Proof:}\\
%
&LHS = i=1 \And n>0 \And x=13\\
%
\myjustification[\Implies]{$A \And B \And C \Implies A \And B$}
&i=1 \And x=13 \\
%
\myjustification[\Implies]{We know that 13 is the 1st emirp from our definition of emirp, also 13$\geq$r in this case}
&i=1 \And Emirp(13,1) \And x=13 \And 13\geq r\\
%
\myjustification[\Implies]{This is our Inv with 13 substituted for x}
&Inv\subst{13}{x}\\
%
\myjustification[\Implies]{clearly}
&RHS\\
\end{align*}

\subsection{\color{blue}Implication 3\color{black}: $\myRefines{r,x:[Inv\subst{13}{r}, Inv]}{r \Ass 13}$}
\begin{align*}
&\text{To prove: } r=r_{0} \And Inv\subst{13}{r} \Implies Inv\subst{13}{r}\\\\
%
&\text{Proof:}\\
%
&LHS = r=r_{0} \And Inv\subst{13}{r}\\
%
\myjustification[\Implies]{$A \And B \Implies A$}
&Inv\subst{13}{r}\\
\myjustification[\Implies]{clearly}
&RHS\\
\end{align*}

\subsection{\color{blue}Implication 4\color{black}: $\myRefines{Inv\And i=n}{Emirp(r,n)}$}
\begin{align*}
&\text{To prove:} Inv\And i=n \Implies Emirp(r,n) \\\\
%
&\text{Proof:}\\
%
&LHS = Inv\And i=n\\
%
\myjustification[\Implies]{Expanding the Invariant}
&Emirp(r,i) \And x\geq r \And i=n\\
%
%
\myjustification[\Implies]{Combining conjuncts}
&Emirp(r,n) \And x\geq r\\
%
%
\myjustification[\Implies]{$A\And B \Implies A$}
&Emirp(r,n)\\
%
\myjustification[\Implies]{Clearly}
&RHS\\
\end{align*}

\subsection{\color{blue}Implication 5\color{black}: $\myRefines{(10)}{skip}$}
To prove skip, we need to prove $pre\Implies post\subst{r_{0}}{r}$
\begin{align*}
&\text{To prove: } Emirp(r, n) \Implies Emirp(r,n)\subst{r_{0}}{r}\\\\
%
&\text{Proof:}\\
%
&LHS = Emirp(r, n)\\
%
\myjustification[\Implies]{Since $r_{0}$ is the value of r in the precondition, $r=r_{0}$ in the precondition}
&Emirp(r_{0}, n)\\
%
\myjustification[\Implies]{clearly}
&RHS\\
\end{align*}


\subsection{\color{blue}Implication 6\color{black}: $\myRefines{[Inv \And i\neq n, Inv\subst{x+1}{x}]}{x\Ass {x+1}}$}
\begin{align*}
&\text{To prove:} [Inv \And i\neq n, Inv\subst{x+1}{x}] \isrefinedby x\Ass {x+1}\\\\
%
&\text{Proof:}\\
%
&LHS = [Inv \And i\neq n, Inv\subst{x+1}{x}]\\
%
\myjustification[\Implies]{Expanding Inv and performing substitution}
&[EMIRP(r,n)\And x\geq r\And i\neq n, EMIRP(r,n)\And x+1\geq r]\\
%
&\text{\bcolor We know that $x\geq r \Implies x+1\geq r$}\\
&\text{\bcolor Since we have not found the $n^{th}$ emirp yet and $x$ is not an emirp, we increment x}\\
%
\myjustification[\Implies]{Therefore, our program can be refined by $x\Ass {x+1}$}
&RHS\\
\end{align*}

\subsection{\color{blue}Implication 7\color{black}: $\myRefines{Inv\subst{x+1}{x}, Inv\subst{x+1}{x}\And a=1}{a \Ass 1}$}
\begin{align*}
&\text{To prove: } a=a_{0} \And Inv\subst{x+1}{x} \Implies (a=1 \And Inv\subst{x+1}{x})\subst{1}{a}\\\\
%
&\text{Proof:}\\
%
&LHS = a=a_{0} \And Inv\subst{x+1}{x}\\
%
\myjustification[\Implies]{1=1 is vacuously true}
&1=1 \And a=a_{0} \And Inv\subst{x+1}{x}\\
%
\myjustification[\Implies]{$A \And B \And C \Implies A\And B$}
&1=1 \And Inv\subst{x+1}{x}\\
%
\myjustification[\Implies]{$1=1 \Implies (a=1 \And Inv\subst{x}{x+1})\subst{1}{i} \text{ (Since, Inv does not involve a)}$}
&(a=1\And Inv\subst{x+1}{x})\subst{1}{a}\\
%
\myjustification[\Implies]{clearly}
&RHS\\
\end{align*}

\subsection{\color{blue}Implication 8\color{black}: $\myRefines{[Inv\subst{x+1}{x} \And a=1,\post(14)]}{[a=1 \And x>0, \post(14)]}$}
\begin{align*}
&\text{To prove: } Inv\subst{x+1}{x} \And a=1 \Implies a=1 \And x>0 \\\\
%
&\text{Proof:}\\
%
&LHS = Inv\subst{x+1}{x} \And a=1\\
%
\myjustification[\Implies]{Expanding Inv and performing substitution}
&Emirp(r,n) \And x+1>=r \And a=1\\
%
\myjustification[\Implies]{Since $x$ and $r$ starts at 13 and we are incrementing x, $x>0$ }
&x>0 \And a=1\\
%
\myjustification[\Implies]{Clearly}
&RHS\\
\end{align*}

\subsection{\color{blue}Implication 9\color{black}: $\myRefines{[a\neq 1 \And \pre(15), \post(15)]}{[[Inv,Inv]]}$}
\begin{align*}
&\text{To prove:} a\neq 1 \And \pre(15) \Implies Inv \\\\
%
&\text{Proof:}\\
%
&LHS = a\neq 1 \And \pre(15)\\
%
\myjustification[\Implies]{$a\neq 1 \Implies$ x is not prime $\Implies$ we have not found a new Emirp}
&EMIRP(r.n)\\
%
%
\myjustification[\Implies]{$x\geq r$ because x only ever increases and started with x=r=13}
&EMIRP(r,n)\And x\geq r\\
%
%
\myjustification[\Implies]{By definition of Inv}
&Inv\\
%
\myjustification[\Implies]{Clearly}
&RHS\\
\end{align*}

\subsection{\color{blue}Implication 10\color{black}: $\myRefines{[Inv, Inv]}{skip}$}
\begin{align*}
&\text{To prove:} Inv \Implies Inv\subst{r_0}{r} \\\\
%
&\text{Proof:}\\
%
&LHS = Inv\\
%
\myjustification[\Implies]{Since $r_0$ is the value of $r$ in the pre-condition, thus in precondition, $r_0 = r$}
&Inv\subst{r_0}{r}\\
%
\myjustification[\Implies]{Clearly}
&RHS\\
\end{align*}

\subsection{\color{blue}Implication 11\color{black}: $\myRefines{(18)}{s \Ass 0}$}
\begin{align*}
&\text{To prove: } s=s_{0} \And (a=1 \text{ }\And \neg\Exi{k \in 2..(x-1)}{x\textbf{ mod }k=0})\Or\\
&((a=0 \And\Exi{k \in 2..(x-1)}{x\textbf{ mod }k=0})) \Implies (s=0 \And x>0)\subst{0}{s}\\\\
%
&\text{Proof:}\\
%
&LHS = s=s_{0} \And ((a=1 \text{ }\And \neg\Exi{k \in 2..(x-1)}{x\textbf{ mod }k=0})\Or\\
&(a=0 \And\Exi{k \in 2..(x-1)}{x\textbf{ mod }k=0}))\\
%
\myjustification[\Implies]{0=0 is vacuously true and $x$ is a positive integer}
&0=0 \And s=s_{0} \And ((a=1 \text{ }\And \neg\Exi{k \in 2..(x-1)}{x\textbf{ mod }k=0})\Or\\
&(a=0 \And \Exi{k \in 2..(x-1)}{x\textbf{ mod }k=0})\And x>0\\
%
\myjustification[\Implies]{$A \And B \And C \Implies A$}
&0=0 \And x>0\\
%
\myjustification[\Implies]{$0=0 \Implies (s=0)\subst{1}{s} \And x>0$ does not involve $s$}
&(s=0 \And x>0)\subst{0}{s}\\
%
\myjustification[\Implies]{clearly}
&RHS\\
\end{align*}


\subsection{\color{blue}Implication 12\color{black}: $\myRefines{[\pre(19), s = \sum\limits_{i=0}^{c(x)}(S_i10^{i})]}{[x = \sum\limits_{i=0}^{c(x)}(S_i10^{(c(x)-i)}) \And x>0, s = \sum\limits_{i=0}^{c(x)}(S_i10^{i})]}$}
\begin{align*}
&\text{To prove: } s=0 \And x>0 \Implies x = \sum\limits_{i=0}^{c(x)}(S_i10^{(c(x)-i)}) \And x>0 \\\\
%
&\text{Proof:}\\
%
&LHS = s=0\And x>0\\
%
\myjustification[\Implies]{$A\And B \Implies A$}
&x>0\\
%
%
\myjustification[\Implies]{$x \in N\And x>0 \Implies x$ can be represented as a sum of $S_{i}10^{c(n)-i}$}
&\text{\bcolor ($S_i$ is a digit from the sequence of digits)}\\ 
&x = \sum\limits_{i=0}^{c(x)}(S_i10^{(c(x)-i)}) \And x>0\\
%
%
\myjustification[\Implies]{Clearly}
&RHS\\
\end{align*}

\subsection{\color{blue}Implication 13\color{black}: $\myRefines{[\pre(21), \pre(21)\And b=1]}{b\Ass1}$}
\begin{align*}
&\text{To prove:} b=b_0 \And \pre(21) \Implies (\pre(21) \And b=1)\subst{1}{b}\\\\
%
&\text{Proof:}\\
%
&LHS = b=b_0 \And \pre(21)\\
%
\myjustification[\Implies]{1=1 is vacously true}
&1=1 \And b=b_0 \And \pre(21)\\
%
\myjustification[\Implies]{$A\And B \And C\Implies A\And B$}
&1=1 \And \pre(21)\\
%
\myjustification[\Implies]{$1=1 \Implies (b=1 \And \pre(21))\subst{1}{b}$ since b does not appear in $\pre(21)$}
&(b=1\And \pre(21))\subst{1}{b}\\
%
\myjustification[\Implies]{Clearly}
&RHS\\
\end{align*}

\subsection{\color{blue}Implication 14\color{black}: $\myRefines{[s = \sum\limits_{i=0}^{c(x)}(S_i10^{i})\And b=1,\post(24)}{[s>0\And b=1, \post(24)]}$}
\begin{align*}
&\text{To prove:} s = \sum\limits_{i=0}^{c(x)}(S_i10^{i})\And b=1 \Implies s>0 \And b=1 \\\\
%
&\text{Proof:}\\
%
&LHS = s = \sum\limits_{i=0}^{c(x)}(S_i10^{i}) \And b=1\\
%
\myjustification[\Implies]{$s = \sum\limits_{i=0}^{c(x)}(S_i10^{i}) \Implies s>0$}
&s>0 \And b=1\\
%
\myjustification[\Implies]{clearly}
&RHS\\
\end{align*}

\subsection{\color{blue}Implication 15\color{black}: $\myRefines{}{}$}
\begin{align*}
&\text{To prove:} BLAH \\\\
%
&\text{Proof:}\\
%
&LHS = BLAH\\
%
\myjustification[\Implies]{BLAH}
&BLAH\\
%
%
\myjustification[\Implies]{BLAH}
&BLAH\\
%
%
\myjustification[\Implies]{BLAH}
&BLAH\\
%
\myjustification[\Implies]{BLAH}
&RHS\\
\end{align*}

\subsection{\color{blue}Implication 16\color{black}: $\myRefines{[Inv\subst{i+1}{i}\subst{x}{r}, Inv]}{i\Ass {i+1}; r\Ass x}$}
\begin{align*}
&\text{To prove:} i=i_0 \And r=r_0\And Inv\subst{i+1}{i}\subst{x}{r} \Implies Inv\subst{i+1}{i}\subst{x}{r} \\\\
%
&\text{Proof:}\\
%
&LHS = i=i_0 \And r=r_0\And Inv\subst{i+1}{i}\subst{x}{r}\\
%
\myjustification[\Implies]{$A\And B\And C \Implies A$}
&Inv\subst{i+1}{i}\subst{x}{r}\\
%
%
\myjustification[\Implies]{Cleary}
&RHS\\
\end{align*}

\subsection{\color{blue}Implication 17\color{black}: $\myRefines{}{}$}
\begin{align*}
&\text{To prove:} BLAH \\\\
%
&\text{Proof:}\\
%
&LHS = BLAH\\
%
\myjustification[\Implies]{BLAH}
&BLAH\\
%
%
\myjustification[\Implies]{BLAH}
&BLAH\\
%
%
\myjustification[\Implies]{BLAH}
&BLAH\\
%
\myjustification[\Implies]{BLAH}
&RHS\\
\end{align*}

\subsection{\color{blue}Implication 18\color{black}: $\myRefines{[Inv,Inv]}{skip}$}
\begin{align*}
&\text{To prove:} Inv \Implies Inv\subst{r_0}{r} \\\\
%
&\text{Proof:}\\
%
&LHS = Inv\\
%
\myjustification[\Implies]{Since $r_0$ is the value of $r$ in the pre-condition, thus in precondition, $r_0 = r$}
&Inv\subst{r_0}{r}\\
%
\myjustification[\Implies]{Clearly}
&RHS\\
\end{align*}

\subsection{\color{blue}Implication 19\color{black}: $\myRefines{[a=1\And r>0, a=1\And r>0\And j=2]}{j\Ass2}$}
\begin{align*}
&\text{To prove:} j=j_0\And a=1\And r>0 \Implies (a=1\And r>0\And j=2)\subst{2}{j} \\\\
%
&\text{Proof:}\\
%
&LHS = j=j_0\And a=1\And r>0\\
%
\myjustification[\Implies]{2=2 is vacously true}
&j=j_0\And a=1\And r>0 \And 2=2\\
%
%
\myjustification[\Implies]{2=2 $\Implies (j=2\And a=1\And r>0)\subst{2}{j}$}
&(j=2\And a=1\And r>0)\subst{2}{j}\\
%
\myjustification[\Implies]{Clearly}
&RHS\\
\end{align*}

\subsection{\color{blue}Implication 20\color{black}: $\myRefines{[a=1\And r>0\And j=2,Inv_2]}{Inv_2,Inv_2}$}
\begin{align*}
&\text{To prove:} a=1\And r>0\And j=2 \Implies Inv_2 \\\\
%
&\text{Proof:}\\
%
&LHS = a=1\And r>0\And j=2\\
%
\myjustification[\Implies]{$A\And B\And C \Implies A$}
&j=2\\
%
%
\myjustification[\Implies]{$\Exi{k\in 2..(2-1)}{\phi}$ is vacously true}
&(a=1 \text{ }\And \neg\Exi{k \in 2..(2-1)}{j\textbf{ mod }k=0}\\
&\Or(a=0 \And \Exi{k \in 2..(2-1)}{j\textbf{ mod }k=0}) \And j=2\\
%
%
\myjustification[\Implies]{combining conjuncts, putting j instead of 2}
&(a=1 \text{ }\And \neg\Exi{k \in 2..(j-1)}{j\textbf{ mod }k=0}\\
&\Or(a=0 \And \Exi{k \in 2..(j-1)}{j\textbf{ mod }k=0})\\
%
\myjustification[\Implies]{By definition of $Inv_2$}
&Inv_2\\
%
\myjustification[\Implies]{Clearly}
&RHS\\
\end{align*}

\subsection{\color{blue}Implication 21\color{black}: $\myRefines{[Inv_2,Inv_2]}{skip}$}
\begin{align*}
&\text{To prove:} Inv_2 \Implies Inv_2\subst{r_0}{r} \\\\
%
&\text{Proof:}\\
%
&LHS = Inv_2\\
%
\myjustification[\Implies]{Since $r_0$ is the value of $r$ in the pre-condition, thus in precondition, $r_0 = r$}
&Inv_2\subst{r_0}{r}\\
%
\myjustification[\Implies]{Clearly}
&RHS\\
\end{align*}

\subsection{\color{blue}Implication 22\color{black}: $\myRefines{[Inv_2 \And j=r, \post(13)]}{[\post(13),\post(13)]}$}
\begin{align*}
&\text{To prove:} Inv_2 \And j=r \Implies post(3)\\\\
%
&\text{Proof:}\\
%
&LHS = (a=1 \text{ }\And \neg\Exi{k \in 2..(j-1)}{j\textbf{ mod }k=0}\\
&\Or(a=0 \And \Exi{k \in 2..(j-1)}{j\textbf{ mod }k=0})\And j=r\\
%
\myjustification[\Implies]{Combining conjuncts}
&(a=1 \text{ }\And \neg\Exi{k \in 2..(r-1)}{r\textbf{ mod }k=0}\\
&\Or(a=0 \And \Exi{k \in 2..(r-1)}{r\textbf{ mod }k=0})\\
%
%
\myjustification[\Implies]{Clearly}
&RHS\\
\end{align*}

\subsection{\color{blue}Implication 23\color{black}: $\myRefines{[\post(3),\post(3)]}{skip}$}
\begin{align*}
&\text{To prove:} \post(3) \Implies \post(3)\subst{r_0}{r} \\\\
%
&\text{Proof:}\\
%
&LHS = \post(3)\\
%
\myjustification[\Implies]{Since $r_0$ is the value of $r$ in the pre-condition, thus in precondition, $r_0 = r$}
&\post(3)\subst{r_0}{r}\\
%
\myjustification[\Implies]{Clearly}
&RHS\\
\end{align*}

\subsection{\color{blue}Implication 24\color{black}: $\myRefines{[Inv_2\subst{j+1}{j}, Inv_2]}{j\Ass{j+1}}$}
\begin{align*}
&\text{To prove:} j=j_0 \And Inv_2\subst{j+1}{j} \Implies Inv_2\subst{j+1}{j}\\\\
%
&\text{Proof:}\\
%
&LHS = j=j_0 \And Inv_2\subst{j+1}{j}\\
%
\myjustification[\Implies]{$A\And B \Implies A$}
&Inv_2\subst{j+1}{j}\\
%
\myjustification[\Implies]{Clearly}
&RHS\\
\end{align*}

\subsection{\color{blue}Implication 25\color{black}: $\myRefines{}{}$}
\begin{align*}
&\text{To prove:} BLAH \\\\
%
&\text{Proof:}\\
%
&LHS = BLAH\\
%
\myjustification[\Implies]{BLAH}
&BLAH\\
%
%
\myjustification[\Implies]{BLAH}
&BLAH\\
%
%
\myjustification[\Implies]{BLAH}
&BLAH\\
%
\myjustification[\Implies]{BLAH}
&RHS\\
\end{align*}

\subsection{\color{blue}Implication 26\color{black}: $\myRefines{[Inv_{2}\subst{j+1}{j]}\subst{0}{a}}{post(11)}$}
\begin{align*}
&\text{To prove:} a=a_0 \And Inv_2\subst{j+1}{j}\subst{0}{a} \Implies \post(11) \\\\
%
&\text{Proof:}\\
%
&LHS = a=a_0 \And Inv_2\subst{j+1}{j}\subst{0}{a}\\
%
\myjustification[\Implies]{Removing conjuncts}
&Inv_2\subst{j+1}{j}\subst{0}{a}\\
%
%
\myjustification[\Implies]{Clearly}
&RHS\\
\end{align*}

\subsection{\color{blue}Implication 27\color{black}: $\myRefines{(12)}{[Inv_{2}, post(11)]}$}
\begin{align*}
&\text{To prove:} BLAH \\\\
%
&\text{Proof:}\\
%
&LHS = BLAH\\
%
\myjustification[\Implies]{BLAH}
&BLAH\\
%
%
\myjustification[\Implies]{BLAH}
&BLAH\\
%
%
\myjustification[\Implies]{BLAH}
&BLAH\\
%
\myjustification[\Implies]{BLAH}
&RHS\\
\end{align*}

\subsection{\color{blue}Implication 28\color{black}: $\myRefines{[Inv_{2}, post(11)]}{skip}$}
\begin{align*}
&\text{To prove:} Inv_2\subst{j+1}{j} \Implies \post(11)\subst{r_0}{r} \\\\
%
&\text{Proof:}\\
%
&LHS = Inv_2\subst{j+1}{j}\\
%
\myjustification[\Implies]{Since $r_0$ is the value of $r$ in the pre-condition, thus in precondition, $r_0 = r$}
&Inv_2\subst{j+1}{j}\subst{r_0}{r}\\
%
\myjustification[\Implies]{$\post(11) = Inv_2\subst{j+1}{j}$}
&RHS\\
\end{align*}


\section{Task 3 - C Code}
\label{sec:task-3}
\lstinputlisting{emirp.c}
\remark{- Write something about how the C code relates.\\- Compare with examples}
\end{document}
