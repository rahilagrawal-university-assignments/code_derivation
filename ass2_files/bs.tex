\documentclass[headings=small,a4paper,12pt]{scrartcl}
% sometimes it's useful to have smaller fractions
\usepackage{nicefrac}
% COMP2111-specific macros. See
% http://www.cse.unsw.edu.au/~cs2111/17s1/LaTeX/primer.html
\usepackage{2111defs2}

\newcommand{\sorted}[3]{\mathit{s'ed}(#1[#2..#3])}
\newcommand{\sort}[3]{\mathit{sort}(#1[#2..#3])}
\newcommand{\pre}{\mathit{pre}}
\newcommand{\post}{\mathit{post}}
\newcommand{\bs}{\textsc{bs}\xspace}

\title{A $\bs$ Derivation}
\author{Kai Engelhardt}
% that one align* may not fit onto a single page
\allowdisplaybreaks

\begin{document}
\maketitle

\section{Introduction}
\label{sec:introduction}

In this short note we're going to derive a recursive implementation of
\emph{binary search} on arrays of integers. The purpose of this note
is not only to demonstrate how to do the derivation but also, by
publishing the \LaTeX~source, to aid students in tyesetting such a
derivation.

This particular version of binary search is supposed to return $-1$ if
the item we're looking for is not in the array, and the index of a
postion of the item if it is.

\section{The Derivation}
\label{sec:derivation}

We start with a spec of the procedure $\bs$. Special care was taken of
the ranges of the left and right boundaries $l$ and $r$ to ensure that
recursive calls satisfy the precondition.
\begin{align*}
  &\PROC~\bs(\VALUE~a, \VALUE~N, \VALUE~v, \VALUE~l, \VALUE~r, \RESULT~p)\cdot{}\\
  &\qquad  \nt{a,N,v,l,r,p:\left[
    \begin{array}{l}
      l\in 0..N\And r\in -1..N-1\And l\leq r+1 \And \sorted a0{N-1},\\
      (a_0[p]=v_0\And p\in l_0..r_0)\Or (p=-1\And\All{x\in l_0..r_0}{a_0[x]\neq v_0})
    \end{array}
  \right]}{(1)}\\
% 
  \lrefstep{(1)}
  {\textbf{c-frame}}
  {
  \nt{p:\left[
  \begin{array}{l}
    l\in 0..N\And r\in -1..N-1\And l\leq r+1 \And \sorted a0{N-1}\\
    (a[p]=v\And p\in l..r)\Or (p=-1\And\All{x\in l..r}{a[x]\neq v})
  \end{array}
  \right]}{(2)}}\\
% 
  \refstep{\textbf{if}}
  {\IF~l=r+1\\
  &\THEN~\nt{p:[l=r+1\And\pre(2),\post(2)]}{(3)}\\
  &\ELSE~\nt{p:[l\neq r+1\And\pre(2),\post(2)]}{(4)}\\
  &\FI}\\
%
  \lrefstep{(3)}
  {\textbf{ass}, justified below in Sect.~\ref{sec:proof3ass}}
  {p\Ass -1}\\
%
  \lrefstep{(4)}
  {\textbf{if}}
  {\IF~a[\nicefrac{(r+l)}2]=v\\
  &\THEN~\nt{p:[a[\nicefrac{(r+l)}2]=v    \And\pre(4),\post(4)]}{(5)}\\
  &\ELSE~\nt{p:[a[\nicefrac{(r+l)}2]\neq v\And\pre(4),\post(4)]}{(6)}\\
  &\FI}\\
%
  \lrefstep{(5)}
  {\textbf{ass}, justified below in Sect.~\ref{sec:proof5ass}}
  {p\Ass \nicefrac{(r+l)}2}\\
%
  \lrefstep{(6)}
  {\textbf{if}}
  {\IF~a[\nicefrac{(r+l)}2]<v\\
  &\THEN~\nt{p:[a[\nicefrac{(r+l)}2]<v     \And\pre(6),\post(6)]}{(7)}\\
  &\ELSE~\nt{p:[a[\nicefrac{(r+l)}2]\not< v\And\pre(6),\post(6)]}{(8)}\\
  &\FI}\\
% 
  \lrefstep{(7)}
  {\textbf{s-post}, justified below in Sect.~\ref{sec:proof7post}}
  {p:[\pre(7),(\post(2))\subst{\nicefrac{(r+l)}2+1}l]}\\
% 
  \refstep
  {\textbf{w-pre}, justified below in Sect.~\ref{sec:proof7pre}}
  {p:[(\pre(1))\subst{\nicefrac{(r+l)}2+1}l,(\post(2))\subst{\nicefrac{(r+l)}2+1}l}\\
% 
  \refstep
  {\textbf{proc}}
  {\bs(a,N,v,\nicefrac{(r+l)}2+1,r,p)}\\
% 
  \lrefstep{(8)}
  {\textbf{s-post}, justified below in Sect.~\ref{sec:proof8post}}
  {p:[\pre(8),(\post(2))\subst{\nicefrac{(r+l)}2-1}r]}\\
% 
  \refstep
  {\textbf{w-pre}, justified below in Sect.~\ref{sec:proof8pre}}
  {p:[(\pre(1))\subst{\nicefrac{(r+l)}2-1}r,(\post(2))\subst{\nicefrac{(r+l)}2-1}r}\\
% 
  \refstep
  {\textbf{proc}}
  {\bs(a,N,v,l,\nicefrac{(r+l)}2-1,p)}
\end{align*}
We gather the code for the procedure body of $\bs$:
\begin{tabbing}% unfortunately, tabbing doesn't do maths
\qquad\=$\IF~l=r+1$\+\\
$\THEN~p\Ass -1$\\
$\ELSE~$\=$\IF~a[\nicefrac{(r+l)}2]=v$\+\\
$\THEN~p\Ass \nicefrac{(r+l)}2$\\
$\ELSE~$\=$\IF~a[\nicefrac{(r+l)}2]<v$\+\\
$\THEN~\bs(a,N,v,\nicefrac{(r+l)}2+1,r,p)$\\
$\ELSE~\bs(a,N,v,l,\nicefrac{(r+l)}2-1,p)$\\
$\FI$\-\\
$\FI$\-\\
$\FI$
\end{tabbing}


\subsection{Proof of $(3)\isrefinedby p\Ass -1$}
\label{sec:proof3ass}

We need to prove validity
\begin{gather*}
  p=p_0\And l=r+1\And\pre(2) \Implies (\post(2))\subst{-1}p
\end{gather*}
i.e., the prerequisite of the relevant instance of \textbf{ass}.
Expanding the definitions and performing the substitution yields
\begin{gather*}
  p=p_0\And {\color{blue}l=r+1}\And l\in 0..N\And r\in -1..N-1\And l\leq r+1 \And
  \sorted a0{N-1}\Implies{}\\
  (a[-1]=v\And -1\in l..r)\Or ({\color{green}-1=-1}\And{\color{blue}\All{x\in l..r}{a[x]\neq v}})
\end{gather*}
Clearly, we should establish the second disjunct of the RHS. Its
{\color{green}first conjunct} is obviously true. And the second is
vacuously true once we take the second conjunct of the LHS, $l=r+1$,
into account, since that means that the range $l..r$ is empty.


\subsection{Proof of $(5)\isrefinedby p\Ass \nicefrac{(r+l)}2$}
\label{sec:proof5ass}

We need to prove validity of
\begin{gather*}
  p=p_0\And\pre(5) \Implies (\post(5))\subst{\nicefrac{(r+l)}2}p
\end{gather*}
i.e., the prerequisite of the relevant instance of \textbf{ass}.
Expanding the definitions and performing the substitution yields
\begin{gather*}
  \left(
    \begin{array}{l}
      p=p_0\And {\color{blue}a[\nicefrac{(r+l)}2]=v} \And {\color{green}l\neq r+1}\And{}\\
      l\in 0..N\And r\in -1..N-1\And {\color{green}l\leq r+1} \And \sorted a0{N-1}
    \end{array}
  \right) \Implies{}\\
  ({\color{blue}a[\nicefrac{(r+l)}2]=v}\And {\color{green}\nicefrac{(r+l)}2\in l..r})\Or
  (\nicefrac{(r+l)}2=-1\And\All{x\in l..r}{a[x]\neq v})\enskip.
\end{gather*}
This time, we should establish the first disjunct of the RHS. The
first conjunct of that is also the second conjunct in the LHS.
Combining the LHS conjuncts $l\neq r+1$ and $l\leq r+1$ we obtain
$l\leq r$ and hence $l\leq \nicefrac{(r+l)}2 \leq r$---the second
conjunct we're trying to establish.


\subsection{Proof of $\pre(7)\And (\post(2))\subst{\nicefrac{(r+l)}2+1}l\Implies \post(7)$}
\label{sec:proof7post}

Expanding the definitions and performing the substitution yields
\begin{gather*}
  \left(
    \begin{array}{l}
      a[\nicefrac{(r+l)}2]<v\And a[\nicefrac{(r+l)}2]\neq v \And l\neq r+1\And{}\\
      l\in 0..N\And r\in -1..N-1\And l\leq r+1 \And \sorted a0{N-1}\And{}\\
      ((a[p]=v\And p\in (\nicefrac{(r+l)}2+1)..r)\Or (p=-1\And\All{x\in (\nicefrac{(r+l)}2+1)..r}{a[x]\neq v}))
    \end{array}
  \right)
  \Implies{}\\
  (a[p]=v\And p\in l..r)\Or (p=-1\And\All{x\in l..r}{a[x]\neq v})
\end{gather*}
According the third line of the LHS, there are two cases to consider:
\begin{enumerate}
\item $a[p]=v\And p\in (\nicefrac{(r+l)}2+1)..r$: the fist disjunct
  of the RHS follows immediately.
\item $p=-1\And\All{x\in (\nicefrac{(r+l)}2+1)..r}{a[x]\neq v}$: we
  prove the second disjunct of the RHS. Its first conjunct is part of
  this case. The second is marginally trickier. To see that the larger
  subrange of $a$ does not contain $v$, we consult the conjuncts
  $a[\nicefrac{(r+l)}2]<v$ and $\sorted a0{N-1}$ of the LHS. Those
  together imply that $v$ cannot be found in
  $a[l..\nicefrac{(r+l)}2]$.
\end{enumerate}

\subsection{Proof of $\pre(7)\Implies (\pre(1))\subst{\nicefrac{(r+l)}2+1}l$}
\label{sec:proof7pre}

Expanding the definitions and performing the substitution yields
\begin{gather*}
  \left(
    \begin{array}{l}
      a[\nicefrac{(r+l)}2]<v\And a[\nicefrac{(r+l)}2]\neq v \And l\neq r+1\And{}\\
      l\in 0..N\And {\color{green}r\in -1..N-1}\And l\leq r+1 \And {\color{blue}\sorted a0{N-1}}
    \end{array}
  \right)
  \Implies{}\\
  \nicefrac{(r+l)}2+1\in 0..N\And {\color{green}r\in -1..N-1}\And \nicefrac{(r+l)}2+1\leq r+1 \And {\color{blue}\sorted a0{N-1}}
\end{gather*}
The uncoloured conjuncts of the LHS imply $r,l\in 0..N-1$ and
$l\leq r$, which is sufficient to ensure the uncoloured conjuncts of
the RHS.

\subsection{Proof of $\pre(8)\And (\post(2))\subst{\nicefrac{(r+l)}2-1}r\Implies \post(8)$}
\label{sec:proof8post}

etc.
\subsection{Proof of $\pre(7)\Implies (\pre(1))\subst{\nicefrac{(r+l)}2-1}r$}
\label{sec:proof8pre}

etc
\end{document}
